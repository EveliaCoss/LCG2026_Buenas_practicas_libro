% Options for packages loaded elsewhere
\PassOptionsToPackage{unicode}{hyperref}
\PassOptionsToPackage{hyphens}{url}
%
\documentclass[
  11pt,
  a4paper,
]{book}

\usepackage{amsmath,amssymb}
\usepackage{iftex}
\ifPDFTeX
  \usepackage[T1]{fontenc}
  \usepackage[utf8]{inputenc}
  \usepackage{textcomp} % provide euro and other symbols
\else % if luatex or xetex
  \usepackage{unicode-math}
  \defaultfontfeatures{Scale=MatchLowercase}
  \defaultfontfeatures[\rmfamily]{Ligatures=TeX,Scale=1}
\fi
\usepackage{lmodern}
\ifPDFTeX\else  
    % xetex/luatex font selection
\fi
% Use upquote if available, for straight quotes in verbatim environments
\IfFileExists{upquote.sty}{\usepackage{upquote}}{}
\IfFileExists{microtype.sty}{% use microtype if available
  \usepackage[]{microtype}
  \UseMicrotypeSet[protrusion]{basicmath} % disable protrusion for tt fonts
}{}
\makeatletter
\@ifundefined{KOMAClassName}{% if non-KOMA class
  \IfFileExists{parskip.sty}{%
    \usepackage{parskip}
  }{% else
    \setlength{\parindent}{0pt}
    \setlength{\parskip}{6pt plus 2pt minus 1pt}}
}{% if KOMA class
  \KOMAoptions{parskip=half}}
\makeatother
\usepackage{xcolor}
\setlength{\emergencystretch}{3em} % prevent overfull lines
\setcounter{secnumdepth}{5}
% Make \paragraph and \subparagraph free-standing
\makeatletter
\ifx\paragraph\undefined\else
  \let\oldparagraph\paragraph
  \renewcommand{\paragraph}{
    \@ifstar
      \xxxParagraphStar
      \xxxParagraphNoStar
  }
  \newcommand{\xxxParagraphStar}[1]{\oldparagraph*{#1}\mbox{}}
  \newcommand{\xxxParagraphNoStar}[1]{\oldparagraph{#1}\mbox{}}
\fi
\ifx\subparagraph\undefined\else
  \let\oldsubparagraph\subparagraph
  \renewcommand{\subparagraph}{
    \@ifstar
      \xxxSubParagraphStar
      \xxxSubParagraphNoStar
  }
  \newcommand{\xxxSubParagraphStar}[1]{\oldsubparagraph*{#1}\mbox{}}
  \newcommand{\xxxSubParagraphNoStar}[1]{\oldsubparagraph{#1}\mbox{}}
\fi
\makeatother


\providecommand{\tightlist}{%
  \setlength{\itemsep}{0pt}\setlength{\parskip}{0pt}}\usepackage{longtable,booktabs,array}
\usepackage{calc} % for calculating minipage widths
% Correct order of tables after \paragraph or \subparagraph
\usepackage{etoolbox}
\makeatletter
\patchcmd\longtable{\par}{\if@noskipsec\mbox{}\fi\par}{}{}
\makeatother
% Allow footnotes in longtable head/foot
\IfFileExists{footnotehyper.sty}{\usepackage{footnotehyper}}{\usepackage{footnote}}
\makesavenoteenv{longtable}
\usepackage{graphicx}
\makeatletter
\newsavebox\pandoc@box
\newcommand*\pandocbounded[1]{% scales image to fit in text height/width
  \sbox\pandoc@box{#1}%
  \Gscale@div\@tempa{\textheight}{\dimexpr\ht\pandoc@box+\dp\pandoc@box\relax}%
  \Gscale@div\@tempb{\linewidth}{\wd\pandoc@box}%
  \ifdim\@tempb\p@<\@tempa\p@\let\@tempa\@tempb\fi% select the smaller of both
  \ifdim\@tempa\p@<\p@\scalebox{\@tempa}{\usebox\pandoc@box}%
  \else\usebox{\pandoc@box}%
  \fi%
}
% Set default figure placement to htbp
\def\fps@figure{htbp}
\makeatother

\usepackage{graphicx}
\usepackage{float}
\makeatletter
\@ifpackageloaded{tcolorbox}{}{\usepackage[skins,breakable]{tcolorbox}}
\@ifpackageloaded{fontawesome5}{}{\usepackage{fontawesome5}}
\definecolor{quarto-callout-color}{HTML}{909090}
\definecolor{quarto-callout-note-color}{HTML}{0758E5}
\definecolor{quarto-callout-important-color}{HTML}{CC1914}
\definecolor{quarto-callout-warning-color}{HTML}{EB9113}
\definecolor{quarto-callout-tip-color}{HTML}{00A047}
\definecolor{quarto-callout-caution-color}{HTML}{FC5300}
\definecolor{quarto-callout-color-frame}{HTML}{acacac}
\definecolor{quarto-callout-note-color-frame}{HTML}{4582ec}
\definecolor{quarto-callout-important-color-frame}{HTML}{d9534f}
\definecolor{quarto-callout-warning-color-frame}{HTML}{f0ad4e}
\definecolor{quarto-callout-tip-color-frame}{HTML}{02b875}
\definecolor{quarto-callout-caution-color-frame}{HTML}{fd7e14}
\makeatother
\makeatletter
\@ifpackageloaded{bookmark}{}{\usepackage{bookmark}}
\makeatother
\makeatletter
\@ifpackageloaded{caption}{}{\usepackage{caption}}
\AtBeginDocument{%
\ifdefined\contentsname
  \renewcommand*\contentsname{Table of contents}
\else
  \newcommand\contentsname{Table of contents}
\fi
\ifdefined\listfigurename
  \renewcommand*\listfigurename{List of Figures}
\else
  \newcommand\listfigurename{List of Figures}
\fi
\ifdefined\listtablename
  \renewcommand*\listtablename{List of Tables}
\else
  \newcommand\listtablename{List of Tables}
\fi
\ifdefined\figurename
  \renewcommand*\figurename{Figure}
\else
  \newcommand\figurename{Figure}
\fi
\ifdefined\tablename
  \renewcommand*\tablename{Table}
\else
  \newcommand\tablename{Table}
\fi
}
\@ifpackageloaded{float}{}{\usepackage{float}}
\floatstyle{ruled}
\@ifundefined{c@chapter}{\newfloat{codelisting}{h}{lop}}{\newfloat{codelisting}{h}{lop}[chapter]}
\floatname{codelisting}{Listing}
\newcommand*\listoflistings{\listof{codelisting}{List of Listings}}
\makeatother
\makeatletter
\makeatother
\makeatletter
\@ifpackageloaded{caption}{}{\usepackage{caption}}
\@ifpackageloaded{subcaption}{}{\usepackage{subcaption}}
\makeatother

\usepackage{bookmark}

\IfFileExists{xurl.sty}{\usepackage{xurl}}{} % add URL line breaks if available
\urlstyle{same} % disable monospaced font for URLs
\hypersetup{
  pdftitle={Introducción a la Bioinformática},
  pdfauthor={Dra. Evelia Coss},
  hidelinks,
  pdfcreator={LaTeX via pandoc}}


\title{Introducción a la Bioinformática}
\author{Dra. Evelia Coss}
\date{2025-08-15}

\begin{document}
\frontmatter
\maketitle

\renewcommand*\contentsname{Table of contents}
{
\setcounter{tocdepth}{2}
\tableofcontents
}

\mainmatter
\bookmarksetup{startatroot}

\chapter*{Información general}\label{informaciuxf3n-general}
\addcontentsline{toc}{chapter}{Información general}

\markboth{Información general}{Información general}

\begin{figure}

\begin{minipage}{0.50\linewidth}
\pandocbounded{\includegraphics[keepaspectratio]{index_files/mediabag/logo-liigh-unam.png}}\end{minipage}%

\end{figure}%

\begin{itemize}
\item
  \subsection{Sobre el curso 📌}

  \begin{itemize}
  \tightlist
  \item
    \textbf{Semestre:} 1ero
  \item
    \textbf{Fechas:} agosto-diciembre
  \item
    \textbf{Lugar:} LIIGH-UNAM
  \item
    \textbf{Duración del curso:} 2 horas por clase
  \end{itemize}

  \subsubsection*{\texorpdfstring{\textbf{Instructores:}}{Instructores:}}\label{instructores}
  \addcontentsline{toc}{subsubsection}{\textbf{Instructores:}}

  \begin{itemize}
  \tightlist
  \item
    \textbf{Jair Santiago García Sotelo} - Técnico académico,
    LIIGH-UNAM.
  \item
    \textbf{Evelia Lorena Coss-Navarrete} - PostDoc, LIIGH-UNAM.
    \href{https://eveliacoss.github.io/}{Pagina web}
  \end{itemize}

  \subsubsection*{\texorpdfstring{\textbf{Resumen:}}{Resumen:}}\label{resumen}
  \addcontentsline{toc}{subsubsection}{\textbf{Resumen:}}

  En este curso se abordarán los conceptos fundamentales para el uso y
  manejo de las herramientas bioinformáticas básicas más relevantes y
  comúnmente empleadas en el área. Se incluirá el manejo adecuado del
  \textbf{sistema operativo Linux, el uso de bases de datos biológicas,
  la manipulación de secuencias, así como la creación de programas con
  funcionalidades sencillas aplicables tanto a la bioinformática como a
  la programación general.}

  \subsubsection*{\texorpdfstring{\textbf{Objetivos:}}{Objetivos:}}\label{objetivos}
  \addcontentsline{toc}{subsubsection}{\textbf{Objetivos:}}

  En esta curso aprenderás a:

  \begin{enumerate}
  \def\labelenumi{\arabic{enumi}.}
  \tightlist
  \item
    Mis primeros pasos en bash.
  \item
    Consultar información detallada sobre archivos y directorios
    utilizando herramientas del sistema operativo Linux.
  \item
    Permisos y como cambiarlos.
  \item
    Diseñar y ejecutar scripts completos en Bash para automatizar tareas
    bioinformáticas.
  \item
    Realizar búsquedas de patrones en secuencias genéticas utilizando
    expresiones regulares y herramientas de línea de comandos.
  \end{enumerate}

  \section*{Citar y reutilizar el material del
  curso}\label{citar-y-reutilizar-el-material-del-curso}
  \addcontentsline{toc}{section}{Citar y reutilizar el material del
  curso}

  \markright{Citar y reutilizar el material del curso}

  Los datos del curso se pueden reutilizar y adaptar libremente con la
  debida atribución. Todos los datos de los cursos de estos repositorios
  están sujetos a la licencia
  \href{https://creativecommons.org/licenses/by-nc-sa/4.0/}{Attribution-NonCommercial-ShareAlike
  4.0 International (CC BY-NC-SA 4.0)}.

  \subsection{Requisitos previos}

  \begin{itemize}
  \tightlist
  \item
    Debes contar con el sistema operativo Linux instalado \textbf{o}
    disponer de una terminal Bash Shell funcional en tu sistema
    operativo. Para más detalles sobre cómo instalar o configurar estos
    componentes, consulta la sección \emph{Instalaciones y
    requerimientos previos}.
  \end{itemize}

  \subsection{Agenda 📆}

  \begin{longtable}[]{@{}
    >{\raggedright\arraybackslash}p{(\linewidth - 2\tabcolsep) * \real{0.9309}}
    >{\centering\arraybackslash}p{(\linewidth - 2\tabcolsep) * \real{0.0691}}@{}}
  \toprule\noalign{}
  \begin{minipage}[b]{\linewidth}\raggedright
  Temas
  \end{minipage} & \begin{minipage}[b]{\linewidth}\centering
  Fechas
  \end{minipage} \\
  \midrule\noalign{}
  \endhead
  \bottomrule\noalign{}
  \endlastfoot
  \begin{minipage}[t]{\linewidth}\raggedright
  \begin{itemize}
  \tightlist
  \item
    \href{https://lcg-cursos.github.io/material/introbioinfo/L3-shell.html\#1}{Introducción
    a la bioinformática}
  \end{itemize}
  \end{minipage} & 12 agosto \\
  \begin{minipage}[t]{\linewidth}\raggedright
  \begin{itemize}
  \tightlist
  \item
    \href{https://lcg-cursos.github.io/material/introbioinfo/L3-shell.html\#20}{Conceptos
    Unix y GNU/Linux} e instalación del sistema operativo
  \end{itemize}
  \end{minipage} & 14 agosto \\
  \begin{minipage}[t]{\linewidth}\raggedright
  \begin{itemize}
  \tightlist
  \item
    Mis primeros pasos en Bash
    (\href{https://lcg-cursos.github.io/material/introbioinfo/L3-shell.html\#70}{archivos}
    y
    \href{https://lcg-cursos.github.io/material/introbioinfo/L3-shell.html\#90}{permisos})
    y creación de scripts completos
  \end{itemize}
  \end{minipage} & 19 y 21 agosto \\
  \begin{minipage}[t]{\linewidth}\raggedright
  \begin{itemize}
  \tightlist
  \item
    Buenas prácticas en Bioinformática
  \end{itemize}
  \end{minipage} & 26 agosto \\
  \begin{minipage}[t]{\linewidth}\raggedright
  \begin{itemize}
  \tightlist
  \item
    Introducción a Markdown
  \end{itemize}
  \end{minipage} & 28 agosto \\
  & \\
  \end{longtable}
\end{itemize}

\bookmarksetup{startatroot}

\chapter{Temario}\label{temario}

El Plan de clases detallado con actividades, recursos, plantillas,
material de lectura esta disponible en pdf. Aquí listamos la
organización de los temas y las habilidades que el alumno debe lograr.

\begin{longtable}[]{@{}
  >{\raggedright\arraybackslash}p{(\linewidth - 2\tabcolsep) * \real{0.5383}}
  >{\centering\arraybackslash}p{(\linewidth - 2\tabcolsep) * \real{0.4617}}@{}}
\toprule\noalign{}
\begin{minipage}[b]{\linewidth}\raggedright
Temas
\end{minipage} & \begin{minipage}[b]{\linewidth}\centering
Objetivo
\end{minipage} \\
\midrule\noalign{}
\endhead
\bottomrule\noalign{}
\endlastfoot
\href{https://lcg-cursos.github.io/material/introbioinfo/L3-shell.html\#1}{Introducción
a la bioinformática} & \begin{minipage}[t]{\linewidth}\centering
\begin{itemize}
\tightlist
\item
  Comprender qué es Unix, sus características principales y su utilidad
  en entornos científicos y técnicos.
\end{itemize}
\end{minipage} \\
\href{https://lcg-cursos.github.io/material/introbioinfo/L3-shell.html\#20}{Conceptos
Unix y GNU/Linux} y
\href{https://lcg-cursos.github.io/material/introbioinfo/L3-shell.html\#1}{protocolos
de internet} & \begin{minipage}[t]{\linewidth}\centering
\begin{itemize}
\tightlist
\item
  Identificar los principales protocolos de intercambio de datos en
  internet y entender su funcionamiento básico.
\item
  Aplicar el protocolo de transferencia de archivos (FTP/SFTP) y acceder
  a servidores remotos de forma segura y eficiente.
\end{itemize}
\end{minipage} \\
Mis primeros pasos en Bash
(\href{https://lcg-cursos.github.io/material/introbioinfo/L3-shell.html\#70}{archivos}
y
\href{https://lcg-cursos.github.io/material/introbioinfo/L3-shell.html\#90}{permisos})
y creación de scripts completos &
\begin{minipage}[t]{\linewidth}\centering
\begin{itemize}
\tightlist
\item
  Comprender la estructura del sistema de archivos en Unix y saber
  desplazarse por él eficientemente.
\item
  Conocer y aplicar comandos básicos para gestionar archivos y
  directorios.
\item
  Identificar los tipos de archivos en Unix, visualizar su contenido y
  entender su manejo.
\item
  Utilizar herramientas para comprimir y descomprimir archivos.
\end{itemize}
\end{minipage} \\
Buenas prácticas en Bioinformática &
\begin{minipage}[t]{\linewidth}\centering
\begin{itemize}
\tightlist
\item
  Aplicar buenas prácticas en bioinformática para asegurar que los
  análisis sean reproducibles, claros y colaborativos, mediante código
  modular, documentación precisa y control de calidad.
\end{itemize}
\end{minipage} \\
Introducción a Markdown & \begin{minipage}[t]{\linewidth}\centering
\begin{itemize}
\item
  Usar Markdown como formato estándar en bioinformática.
\item
  Comprender las fases del análisis de datos bioinformáticos.
\end{itemize}
\end{minipage} \\
& \\
& \\
\end{longtable}

\bookmarksetup{startatroot}

\chapter{Buenas prácticas en
bioinformática}\label{buenas-pruxe1cticas-en-bioinformuxe1tica}

Notas personales recabadas a partir de los tutoriales y ejemplos 😊.
Espero que les funcione 💜

\begin{tcolorbox}[enhanced jigsaw, arc=.35mm, coltitle=black, left=2mm, colbacktitle=quarto-callout-note-color!10!white, breakable, toptitle=1mm, colback=white, opacitybacktitle=0.6, bottomtitle=1mm, leftrule=.75mm, colframe=quarto-callout-note-color-frame, rightrule=.15mm, title=\textcolor{quarto-callout-note-color}{\faInfo}\hspace{0.5em}{Note}, titlerule=0mm, opacityback=0, toprule=.15mm, bottomrule=.15mm]

Presentación completa empleando Rmarkdown

\end{tcolorbox}

\section{Materiales informativos}\label{materiales-informativos}

\begin{itemize}
\tightlist
\item
  \href{https://comunidadbioinfo.github.io/cdsb2023/creaci\%C3\%B3n-de-vi\%C3\%B1etas.html}{Curso
  de Joselyn Cristina Chávez Fuentes}
\item
  Me ayudo mucho este
  \href{https://www.youtube.com/watch?v=7ZgZ6qUKZvE&ab_channel=DaniMedi}{Video}
\item
  \href{https://comunidadbioinfo.github.io/cdsb2023/documentaci\%C3\%B3n-de-funciones.html}{Documentación
  de funciones de Andrés Arredondo Cruz}
\end{itemize}

💪 Estuve muy intensa viendo su codigo. Muchas gracias por tenerlos
publico.

\section{\texorpdfstring{\textbf{Un algoritmo nos permite resolver un
problema
⭐}}{Un algoritmo nos permite resolver un problema ⭐}}\label{un-algoritmo-nos-permite-resolver-un-problema}

Un \textbf{algoritmo} es un método para resolver un problema mediante
una serie de pasos \textbf{definidos, precisos} y \textbf{finitos}.

\begin{itemize}
\tightlist
\item
  \textbf{Definido}: si se sigue dos veces, se obtiene el mismo
  resultado. Es reproducible.
\item
  \textbf{Preciso}: implica el orden de realización de cada uno de los
  pasos.
\item
  \textbf{Finito}: Tiene un numero determinado de pasos, implica que
  tiene un fin.
\end{itemize}

\begin{quote}
Un algoritmo podemos definirlo como un \textbf{programa o software}.
\end{quote}

\href{https://allisonhorst.com/allison-horst}{\pandocbounded{\includegraphics[keepaspectratio]{reproducible/figures/allison-horst-code-kitchen.png}}}

\section{\texorpdfstring{\textbf{Para escribir un buen software
necesitas:}}{Para escribir un buen software necesitas:}}\label{para-escribir-un-buen-software-necesitas}

\begin{quote}
Escribir \textbf{código mantenible (maintainable code), usar control de
versiones (version control) y rastreadores de problemas (issue
trackers), revisiones de código (code reviews), pruebas unitarias (unit
testing) y automatización de tareas (task automation)}.

\href{https://journals.plos.org/plosbiology/article?id=10.1371/journal.pbio.1001745}{Wilson,
\emph{et al.} 2014. \emph{PLOS Biology}}
\end{quote}

En bioinformática, es fundamental garantizar el uso \textbf{ético y
responsable de datos sensibles}, como los genomas humanos, respetando la
privacidad y los marcos legales vigentes. Al mismo tiempo, se debe
fomentar la ciencia abierta \textbf{mediante prácticas transparentes y
reproducibles}, sin comprometer la integridad de la información. Estas
acciones no solo fortalecen la \textbf{confianza en los resultados},
sino que también previenen errores graves que podrían derivar en la
\textbf{retracción de artículos científicos}.

\href{https://devcom.com/tech-blog/code-quality-definition-how-to-improve-code-quality/}{\begin{center}
\pandocbounded{\includegraphics[keepaspectratio]{reproducible/figures/what-code-quality-1.jpg}}
\end{center}
}

\begin{tcolorbox}[enhanced jigsaw, arc=.35mm, coltitle=black, left=2mm, colbacktitle=quarto-callout-note-color!10!white, breakable, toptitle=1mm, colback=white, opacitybacktitle=0.6, bottomtitle=1mm, leftrule=.75mm, colframe=quarto-callout-note-color-frame, rightrule=.15mm, title=\textcolor{quarto-callout-note-color}{\faInfo}\hspace{0.5em}{Pasos para escribir un buen software}, titlerule=0mm, opacityback=0, toprule=.15mm, bottomrule=.15mm]

\begin{enumerate}
\def\labelenumi{\arabic{enumi}.}
\item
  Análisis del problema / Definir el problema
\item
  Diseño del algoritmo / Diseño del programa
\item
  Codificación / Escribir el código
\item
  Compilación y ejecución del programa
\item
  Verificación / Realizar pruebas
\item
  Depuración / Detectar los errores y corregirlos
\end{enumerate}

\begin{quote}
Programacion defensiva
\end{quote}

\begin{enumerate}
\def\labelenumi{\arabic{enumi}.}
\setcounter{enumi}{6}
\tightlist
\item
  Documentación
\end{enumerate}

\end{tcolorbox}

\begin{center}
\pandocbounded{\includegraphics[keepaspectratio]{index_files/mediabag/homersapien.jpg}}
\end{center}

\section{\texorpdfstring{\textbf{Paso 7:
Documentación}}{Paso 7: Documentación}}\label{paso-7-documentaciuxf3n}

\begin{tcolorbox}[enhanced jigsaw, arc=.35mm, coltitle=black, left=2mm, colbacktitle=quarto-callout-note-color!10!white, breakable, toptitle=1mm, colback=white, opacitybacktitle=0.6, bottomtitle=1mm, leftrule=.75mm, colframe=quarto-callout-note-color-frame, rightrule=.15mm, title=\textcolor{quarto-callout-note-color}{\faInfo}\hspace{0.5em}{Note}, titlerule=0mm, opacityback=0, toprule=.15mm, bottomrule=.15mm]

\begin{itemize}
\item
  \emph{Título} (opcional)
\item
  \emph{Autor (author)}: Su nombre
\item
  \emph{Dia (date)}: Fecha de creación
\item
  \emph{Paquetes (packages)}
\item
  \emph{Directorio de trabajo (Working directory)}: En que carpeta se
  encuentra tu datos y programa.
\item
  \emph{Información descriptiva del programa (Description)}: ¿Para qué
  sirve el programa? Ej: El siguiente programa realiza la suma de dos
  numeros enteros a partir de la entrada del usuario y posteriormente la
  imprime en pantalla.
\item
  \emph{Usage} ¿Cómo se utiliza?
\item
  \emph{Argumentos (Arguments)}

  \begin{itemize}
  \item
    \emph{Información de entrada (Data Inputs)}: Ej: Solo numeros
    enteros (sin decimales).
  \item
    \emph{Información de salida (Outpus)}: Graficas, figuras, tablas,
    etc.
  \end{itemize}
\end{itemize}

\end{tcolorbox}

\begin{center}
\pandocbounded{\includegraphics[keepaspectratio]{index_files/mediabag/meme_documentacion.jpg}}
\end{center}

\section{Puntos claves para buenas prácticas en bioinfo
⭐}\label{puntos-claves-para-buenas-pruxe1cticas-en-bioinfo}

\begin{enumerate}
\def\labelenumi{\arabic{enumi}.}
\item
  Escriba \textbf{programas para personas, no para computadoras}
  (Documenta qué hace y por qué). - Se coherente en la nomenclatura,
  indentación y otros aspectos del estilo.
\item
  Modularidad: Divide los programas en \emph{funciones cortas de un solo
  propósito.} 💻 📚
\item
  \textbf{No repitas tu código}. Crea pasos reproducibles o que se
  repitan por si solas. ➰
\item
  Planifique los errores (\textbf{Programacion defensiva}) 🚩
\item
  Optimice el software sólo después de que funcione correctamente. - Si
  funciona no lo modifiques, simplificalo.
\item
  Colaborar - Busque siempre bibliotecas de software bien mantenidas que
  hagan lo que necesita. 👥
\end{enumerate}

\begin{tcolorbox}[enhanced jigsaw, arc=.35mm, coltitle=black, left=2mm, colbacktitle=quarto-callout-note-color!10!white, breakable, toptitle=1mm, colback=white, opacitybacktitle=0.6, bottomtitle=1mm, leftrule=.75mm, colframe=quarto-callout-note-color-frame, rightrule=.15mm, title=\textcolor{quarto-callout-note-color}{\faInfo}\hspace{0.5em}{Ejemplo de como realizo mis documentos 💜}, titlerule=0mm, opacityback=0, toprule=.15mm, bottomrule=.15mm]

Aqui les dejo el script que les doy a mis alumnos
\href{https://github.com/EveliaCoss/RNAseq_classFEB2024/blob/main/Practica_Dia3/scripts/VisualizacionDatos.R}{VisualizacionDatos.R}
del curso de
\href{https://github.com/EveliaCoss/RNAseq_classFEB2024}{Análisis de
datos de RNA-Seq}.

\end{tcolorbox}

\section{Referencias}\label{referencias}

\begin{itemize}
\tightlist
\item
  Presentación de
  \href{https://miriamll.github.io/teaching_R_Rmd/Repro\#1}{Reproducibilidad
  de Miriam Lerma}
\item
  Presentación
  \href{https://miriamll.github.io/teaching_R_Rmd/GitGithubZenodo\#1}{introducción
  a Git, GitHub y Zenodo de Miriam Lerma}
\item
  \href{https://lcg-cursos.github.io/material/introbioinfo/L2-buenas-practicas.html\#1}{Presentación
  de Introducción a la Bioinformática - Heladia Salgado}
\item
  \href{https://flor14.github.io/rladies-jujuy/presentacion.html?panelset=compendio&panelset1=bibliograf\%25C3\%25ADa\#3}{Mi
  próximo artículo científico en R de Florencia D´Andrea}
\item
  \href{https://github.com/WhitakerLab/ReproducibleResearch}{Kristie
  Reproducible Research}
\item
  \href{https://figshare.com/articles/journal_contribution/Showing_your_working_a_how_to_guide_to_reproducible_research/5443201/1?file=9410686}{Presentación
  de Kristie}
\item
  \href{https://ifb-elixirfr.github.io/IFB-FAIR-bioinfo-training/assets/pdf/Session2020/01_introduction.pdf}{FAIR-
  bioinformatic}
\item
  \href{https://github.com/AliciaMstt/BioinfInvRepro2016-II/blob/master/Unidad1/Unidad1_Bioinf_e_Investigaci\%C3\%B3n_Reproducible.md}{Unidad
  1 Bioinformática e investigación reproducible}
\item
  \href{https://academic.oup.com/bib/article/24/6/bbad375/7326135}{The
  five pillars of computational reproducibility: bioinformatics and
  beyond}
\item
  \href{https://open-science-training-handbook.github.io/Open-Science-Training-Handbook_ES/02OpenScienceBasics/04ReproducibleResearchAndDataAnalysis.html}{Investigación
  reproducible y análisis de datos}
\item
  Haydee tutorial:
  \href{https://haydeeperuyero.github.io/Computo_Cientifico/}{Temas
  Selectos de Análisis Numérico y Computación Científica: Computo
  científico para el análisis de datos}
\item
  Alejandra Medina tutorial:
  \href{https://comunidadbioinfo.github.io/cdsb2023/control-de-versiones-con-github-y-rstudio.html}{Control
  de versiones con GitHub y RStudio}
\item
  Wilson, et al.~2014.
  \href{https://journals.plos.org/plosbiology/article?id=10.1371/journal.pbio.1001745}{Best
  Practices for Scientific Computing}. PLOS Biology
\item
  Evelia Coss - tutorial
  \href{https://github.com/EveliaCoss/Buenaspracticas_R_Mayo2024}{Buenas
  practicas en R}
\item
  Evelia Coss - \href{https://github.com/EveliaCoss/Make_yourCV}{Make
  your CV tutorial}
\item
  Markdwn Guide -
  \href{https://www.markdownguide.org/getting-started/}{Getting Started}
\item
  Allison Horst -
  \href{https://allisonhorst.com/allison-horst}{Imagenes}
\item
  \href{https://srvanderplas.github.io/stat-computing-r-python/}{Statistical
  Computing using R and Python}
\item
  \href{https://learning.nceas.ucsb.edu/2024-10-coreR/}{Training
  Materials}
\item
  \href{https://damiandeluca.com.ar/como-comenzar-con-visual-studio-code-una-guia-para-principiantes}{Visual
  Studio Code}
\item
  \href{https://blog.ml.cmu.edu/2020/08/31/5-reproducibility/}{Articulo
  reproducibilidad}
\end{itemize}

\bookmarksetup{startatroot}

\chapter{Introducción a Markdown}\label{introducciuxf3n-a-markdown}

\section{¿Qué es Markdown?}\label{quuxe9-es-markdown}

Markdown es un \textbf{lenguaje de marcado ligero} que permite añadir
elementos de formato a documentos de \emph{texto plano}. Creado por John
Gruber en 2004, Markdown es actualmente uno de los lenguajes más
populares del mundo.

\begin{itemize}
\tightlist
\item
  Sencillo y legible, sin necesidad de usar editores complicados o HTML.
\item
  \textbf{lenguaje de marcado ligero:} Escribir texto que incluye
  instrucciones simples para darle formato.
\end{itemize}

\begin{tcolorbox}[enhanced jigsaw, arc=.35mm, coltitle=black, left=2mm, colbacktitle=quarto-callout-note-color!10!white, breakable, toptitle=1mm, colback=white, opacitybacktitle=0.6, bottomtitle=1mm, leftrule=.75mm, colframe=quarto-callout-note-color-frame, rightrule=.15mm, title=\textcolor{quarto-callout-note-color}{\faInfo}\hspace{0.5em}{🧩 ¿Por qué se llama ``ligero''?}, titlerule=0mm, opacityback=0, toprule=.15mm, bottomrule=.15mm]

Porque:

\begin{itemize}
\tightlist
\item
  Usa símbolos sencillos como \texttt{*,\ \#,} o \texttt{-} para indicar
  formato.
\item
  No requiere cerrar etiquetas como
  \texttt{\textless{}strong\textgreater{}\ o\ \textless{}div\textgreater{}}
  (como en HTML).
\item
  Es minimalista y limpio, ideal para escribir rápido sin distracciones.
\end{itemize}

\end{tcolorbox}

\section{🧠 ¿Para qué se usa?}\label{para-quuxe9-se-usa}

\begin{itemize}
\tightlist
\item
  Crear documentación técnica o científica
\item
  Escribir \textbf{README.md} en proyectos de GitHub
\item
  Generar contenido para sitios web
\item
  Crear
  \href{https://quarto.org/docs/get-started/hello/rstudio.html}{tesis},
  notas, \href{https://bookdown.org/yihui/rmarkdown-cookbook/}{libros},
  \href{https://bookdown.org/yihui/rmarkdown/xaringan.html}{presentaciones}
  y posters
\end{itemize}

\textbf{En R}

\begin{longtable}[]{@{}
  >{\raggedright\arraybackslash}p{(\linewidth - 2\tabcolsep) * \real{0.5000}}
  >{\raggedright\arraybackslash}p{(\linewidth - 2\tabcolsep) * \real{0.5000}}@{}}
\toprule\noalign{}
\begin{minipage}[b]{\linewidth}\raggedright
Paquete / Programa
\end{minipage} & \begin{minipage}[b]{\linewidth}\raggedright
Uso principal
\end{minipage} \\
\midrule\noalign{}
\endhead
\bottomrule\noalign{}
\endlastfoot
\href{https://github.com/EveliaCoss/RmarkdownGraphs_notes}{\texttt{rmarkdown}}
& Crear documentos reproducibles con texto, código y resultados \\
\texttt{knitr} & Ejecutar chunks de código en RMarkdown \\
\href{https://quarto.org/docs/get-started/hello/vscode.html}{\texttt{Quarto}}
& Plataforma moderna para documentos con R, Python, etc. \\
\texttt{bookdown} & Crear libros técnicos con Markdown + RMarkdown \\
\texttt{blogdown} & Generar blogs científicos usando Markdown y Hugo \\
\texttt{flexdashboard} & Dashboards interactivos escritos en Markdown \\
\end{longtable}

\textbf{En Python}

\begin{longtable}[]{@{}
  >{\raggedright\arraybackslash}p{(\linewidth - 2\tabcolsep) * \real{0.5000}}
  >{\raggedright\arraybackslash}p{(\linewidth - 2\tabcolsep) * \real{0.5000}}@{}}
\toprule\noalign{}
\begin{minipage}[b]{\linewidth}\raggedright
Paquete / Programa
\end{minipage} & \begin{minipage}[b]{\linewidth}\raggedright
Uso principal
\end{minipage} \\
\midrule\noalign{}
\endhead
\bottomrule\noalign{}
\endlastfoot
Jupyter Notebooks & Celdas Markdown para explicar y documentar junto con
código \\
\texttt{Quarto} & Igual que en R, permite combinar Markdown con código
Python \\
\texttt{mkdocs} & Generar sitios web de documentación técnica desde
archivos Markdown \\
\texttt{Sphinx\ +\ MyST} & Documentación avanzada con soporte para
Markdown \\
\texttt{markdown}, \texttt{mistune} & Librerías para convertir Markdown
a HTML \\
\texttt{nbconvert} & Exportar notebooks con Markdown a PDF, HTML, LaTeX,
etc. \\
\end{longtable}

\section{¿Cómo funciona?}\label{cuxf3mo-funciona}

Pasos generales:

\begin{enumerate}
\def\labelenumi{\arabic{enumi}.}
\tightlist
\item
  Crear un archivo Markdown con un editor de texto o una aplicación
  específica. El archivo debe tener la extensión
  \texttt{.md\ o\ .markdown}.
\item
  Se abre el archivo Markdown en una aplicación Markdown.
\item
  Se utiliza la aplicación Markdown para convertir el archivo Markdown
  en un documento HTML.
\item
  Visualización del archivo HTML en un navegador web o use la aplicación
  Markdown para convertirlo a otro formato de archivo, como PDF.
\end{enumerate}

\begin{tcolorbox}[enhanced jigsaw, arc=.35mm, coltitle=black, left=2mm, colbacktitle=quarto-callout-note-color!10!white, breakable, toptitle=1mm, colback=white, opacitybacktitle=0.6, bottomtitle=1mm, leftrule=.75mm, colframe=quarto-callout-note-color-frame, rightrule=.15mm, title=\textcolor{quarto-callout-note-color}{\faInfo}\hspace{0.5em}{Note}, titlerule=0mm, opacityback=0, toprule=.15mm, bottomrule=.15mm]

La aplicación y el procesador de Markdown son dos componentes
independientes. Para simplificar, los he combinado en un solo elemento
(``aplicación de Markdown'') en la figura a continuación.

\end{tcolorbox}

\begin{figure}[H]

{\centering \pandocbounded{\includegraphics[keepaspectratio]{reproducible/figures/markdown-flowchart.png}}

}

\caption{Imagen tomada de:
https://www.markdownguide.org/getting-started/}

\end{figure}%

\section{\texorpdfstring{\textbf{Sintaxis de escritura y formato
básicos}}{Sintaxis de escritura y formato básicos}}\label{sintaxis-de-escritura-y-formato-buxe1sicos}

Revisar la
\href{https://docs.github.com/es/get-started/writing-on-github/getting-started-with-writing-and-formatting-on-github/basic-writing-and-formatting-syntax}{documentación
de GitHub}

\section{Ejercicio individual: Página de inicio en
GitHub}\label{ejercicio-individual-puxe1gina-de-inicio-en-github}

\begin{tcolorbox}[enhanced jigsaw, arc=.35mm, coltitle=black, left=2mm, colbacktitle=quarto-callout-note-color!10!white, breakable, toptitle=1mm, colback=white, opacitybacktitle=0.6, bottomtitle=1mm, leftrule=.75mm, colframe=quarto-callout-note-color-frame, rightrule=.15mm, title=\textcolor{quarto-callout-note-color}{\faInfo}\hspace{0.5em}{Actividad}, titlerule=0mm, opacityback=0, toprule=.15mm, bottomrule=.15mm]

\begin{enumerate}
\def\labelenumi{\arabic{enumi}.}
\item
  En GitHub, crea un repositorio con el mismo nombre de usuario.
  Ejemplo: Mi usuario es EveliaCoss y mi repositorio se llama
  EveliaCoss, dandome la siguiente liga
  https://github.com/EveliaCoss/EveliaCoss
\item
  Agrega un README al repositorio con tu informacion. Por Default ya
  trae una plantilla, modificala y adaptala a tu gusto.
\item
  Visualiza como cambia tu inicio en el Github.
\end{enumerate}

\end{tcolorbox}

\href{https://github.com/EveliaCoss}{\pandocbounded{\includegraphics[keepaspectratio]{reproducible/figures/EveliaCoss_inciio.jpg}}}

\section{Referencias}\label{referencias-1}

\begin{itemize}
\tightlist
\item
  \href{https://docs.github.com/es/get-started/writing-on-github/getting-started-with-writing-and-formatting-on-github/basic-writing-and-formatting-syntax}{Documentación
  de GitHub}
\item
  \href{https://michael-franke.github.io/intro-data-analysis/ch-01-01-Rmarkdown.html}{Libro
  de Rmarkdown}
\item
  \href{https://www.markdowntutorial.com/es/lesson/1/}{Ejercicios con
  Markdown}
\item
  \href{https://lcg-cursos.github.io/material/introbioinfo/}{Curso de
  Heladia Salgado}
\item
  Curso de
  \href{https://eveliacoss.github.io/Workshop_GitGithub2025/Parte2.html}{Git
  y GitHub Evelia Coss}
\item
  \href{https://faculty.washington.edu/otoomet/info201-book/markdown.html}{Markdown
  and rmarkdown}
\item
  \href{https://rmarkdown.rstudio.com/gallery.html}{Galeria de Markdown}
\item
  \href{https://zsmith27.github.io/rmarkdown_crash-course/lesson-3-basic-syntax.html}{Lesson
  3: Basic Syntax}
\item
  \href{https://carpentries-incubator.github.io/open-science-with-r/aio/index.html}{Introduction
  to Open Data Science with R}
\item
  \href{https://bookdown.org/hneth/ds4psy/}{Data Science for
  Psychologists}
\end{itemize}


\backmatter


\end{document}
